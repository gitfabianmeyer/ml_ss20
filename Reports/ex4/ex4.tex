\documentclass[a4paper,parskip=half-]{scrartcl}
\usepackage[utf8]{inputenc}
\usepackage[english]{babel}
\usepackage{fancyhdr}
\usepackage{graphicx}
\usepackage{hyperref}
\usepackage{amsmath}
\usepackage{tabu}
\graphicspath{{c://users//fmeyer//git//ml_ss20//Exercise3//}}
\pagestyle{fancy}
\fancyhf{}
\rhead{
	\small 
		\begin{tabular}[t]{l}
		\textsf{Fabian Meyer, 6816480}\\
		\textsf{Theresa Naß, 7122946}\\
		\end{tabular}}
\lhead{Machine Learning}
\cfoot{	\url{https://github.com/gitfabianmeyer/ml_ss20}}



\begin{document}
	\section*{Support Vector Machines}
	This exercise is about classifying images of Chagas Parasites, similar to exercise 3. Thus, we use the same features as before (i.e. min/max/mean of the color channels).\\
	To find the best hyperparameters, we use Scikit Learn's GridSearchCV. As hyperparameter, we use all combinations of
	\begin{align*}
		C &\in [0.001,0.005,0.01,0.05, 0.1,0.5, 1, 5, 10]\\
		\gamma & \in [1e-09, 2.15443469e-08,4.6415888e-07,	1e-05,\\ &0.000215443469,0.004641588,0.1,2.15443469,46.4158883,1000.0]\\
		Kernels & \in [sigmoid, \,radial\, basis\, kernels,\, linear,\, polynomial \,(d = 1, 2, 3, 4)]
	\end{align*}
	
	\noindent We utilize a 	5-fold cross-validation, which is standard in Scikit's library which is used to to generalize the results of the training, as we use, in contrast to the "traditional" way using a 70/20/10 split, splits of the complete data set to validate the model. This results in a longer training time, which is no problem in the our case, as we only have 60 training instances (We test five folds for 1440 parameter combinations, so 7200 fits in total). A negative aspect of the k-fold is, that we can only use $4/5$ of our data to fit the parameter to our data.
	\subsection*{Results}
	
	In Table \ref{tab1} you can see the best results with the respective mean test score (averaged from the 5 folds). Many different combinations averaged similar results. For the full results see \url{https://github.com/gitfabianmeyer/ml_ss20/blob/master/Exercise4/results.csv}. The SVM algorithm classifies the given problem with all kernel function very good except for the sigmoid kernel, so we are not able to determine a best set of hyperparameter or features.
	
	
	\begin{table}[h!]
		\centering
		\begin{tabular}{|c|c|c|c|c|c|}
			\hline
			C     & Degree & Gamma       & Kernel & Mean Test Score & Rank \\ \hline
			0.005 & 2      & 2.15443e-08 & linear & 0.933333        & 1    \\ \hline
			0.01  & 3      & 0.000215443 & poly   & 0.933333        & 1    \\ \hline
			0.005 & 2      & 46.4159     & linear & 0.933333        & 1    \\ \hline
			0.1   & 2      & 0.000215443 & poly   & 0.933333        & 1    \\ \hline
			0.005 & 4      & 2.15443     & linear & 0.933333        & 1    \\ \hline
			0.005 & 3      & 4.64159e-07 & linear & 0.933333        & 1    \\ \hline
			0.005 & 2      & 1e-09       & linear & 0.933333        & 1    \\ \hline
			0.005 & 4      & 1e-09       & linear & 0.933333        & 1    \\ \hline
			0.005 & 4      & 46.4159     & linear & 0.933333        & 1    \\ \hline
			0.005 & 1      & 1000        & linear & 0.933333        & 1    \\ \hline
		\end{tabular}
	\caption{Top ten results with hyperparameter}
	\label{tab1}
	\end{table}

	Due to the design of Scikit's GridSearchCV module, we are not able to report the number of support vectors for every set of parameters, but for the best set of parameter for each kernel. Table \ref{tab2} shows the number of support vectors for each kernel. It is observable, that the number of support vectors increases when the score decreases.
	
	\begin{table}[h!]
		\centering
		\begin{tabular}{|c|c|c|c|c|}
			\hline
			Kernel  & Score & $\gamma$        & C     & Support Vectors        \\ \hline
			Linear  & $0.933$ & $1*10^{-9}$ & $0.01$  & $9\, 8$   \\ \hline
			Poly    & $0.933$ & $0.0046$                       & $0.001$ & $9\, 9$   \\ \hline
			RBF     & $0.916$ & $0.0002$                       & $5$     & $12\, 11$ \\ \hline
			Sigmoid & $0.816$ & $1*10^{-9}$ & $0.001$ & $30\, 30$ \\ \hline
		\end{tabular}
	\caption{Best results for each kernel}
	\label{tab2}
	\end{table}
	
	\section{Conclusion}
	Even though we were able to classify the given problem better (i.e. with a higher accouracy) with the GDA in the last exercise, we would argue that this happened because of the design of the classification process: As we have only 60 training instances, we fit the whole data to the GDA model. Thus, it is very likely that the models suffered from severe overfitting. As we use a five fold cross-validation, we use just 12 examples for testing. In these few examples, outlier will have a great impact on the performance and thus on the mean average score. Thus we can not conclude, that the SVM is less suitable for the given problem than the GDA from exercise 3.
	

\end{document}
